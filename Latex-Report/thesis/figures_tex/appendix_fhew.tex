\begin{table}[!h]
\centering
\caption{Non-recursive 1-D FFT}
\label{fhewcode1:fft1}
\begin{tabular}{l}
\toprule
\begin{lstlisting}[columns=fullflexible, language=C++,escapechar=\$]
void FFT(complex_t input[2*N], complex_t output[N])
{
    int k, n;
    complex_t mul_res, conv_res, add_res;
    // Compute N DFTs of length 1 using naive method
    FFT_label0:for (k = 0; k < N; k++)
    {
       conv_res.re = 1;
       conv_res.im = 0; // -2*PI*n*k2/N2 --> k2 = 0, n = 0 for 1D Convolution
       multiply(input[k], conv_res, &mul_res);
       /* Multiply by the twiddle factors  ( e^(-2*pi*j/N * k1*k2)) and transpose
          conv_from_polar(1, -2.0*PI*k1*k2/N, &conv_res); --> k2 = 0 for 1D Convolution */
       multiply(conv_res, mul_res, &mul_res);
       input[k] = mul_res;
    }

    for(k = 0; k < N; k++)
    {
       output[k].re = 0.0;
       output[k].im = 0.0;
       FFT_label2:for(n = 0; n < N; n++)
       {
    	  std::complex<double> t = std::polar(1.0, -2 * PI * n* k / N);
    	  conv_res.re = std::real(t);
    	  conv_res.im = std::imag(t);
          multiply(input[n], conv_res, &mul_res);
          add(output[k], mul_res, &add_res);
          output[k] = add_res;
       }
    }
 }
//-------------------Utility Functions-------------------------
 void add(complex_t left, complex_t right, complex_t* result)
 {
    (*result).re = left.re + right.re;
    (*result).im = left.im + right.im;
 }

 void multiply(complex_t left, complex_t right, complex_t* result)
 {
    (*result).re = left.re*right.re - left.im*right.im;
    (*result).im = left.re*right.im + left.im*right.re;
 }
\end{lstlisting}
\\
\bottomrule
\end{tabular}
\end{table}

\begin{table}[!h]
\centering
\caption{Radix-2 DIF FFT}
\label{fhewcode2:CTfft}
\begin{tabular}{l}
\toprule
\begin{lstlisting}[columns=fullflexible, language=C++,escapechar=\$]
void fft(Complex x[2*N])
{
	unsigned int k = N, n;
	double thetaT = 3.14159265358979323846264338328L / N;
	Complex phiT = Complex(cos(thetaT), sin(thetaT)), T;

	fft_label0:for(unsigned int i=0; i < 10; i++) 
//because 2^10 = 1024 - After 9 right shifts, k becomes <= 1 => 2^0 = 1
	{
		n = k;
		k >>= 1;
		phiT = phiT * phiT;
		T = 1.0L;
		fft_label1:for (unsigned int l = 0; l < k; l++)
		{
			fft_label2:for (unsigned int a = l; a < N; a += n)
			{
				unsigned int b = a + k;
				Complex t = x[a] - x[b];
				x[a] = x[a] + x[b];
				x[b] = t * T;
			}
			T *= phiT;
		}
	}

	// Decimate
	unsigned int m = (unsigned int)log2(N);
	fft_label3:for (unsigned int a = 0; a < N; a++)
	{
		unsigned int b = a;
		// Reverse bits
		b = (((b & 0xaaaaaaaa) >> 1) | ((b & 0x55555555) << 1));
		b = (((b & 0xcccccccc) >> 2) | ((b & 0x33333333) << 2));
		b = (((b & 0xf0f0f0f0) >> 4) | ((b & 0x0f0f0f0f) << 4));
		b = (((b & 0xff00ff00) >> 8) | ((b & 0x00ff00ff) << 8));
		b = ((b >> 16) | (b << 16)) >> (32 - m);
		if (b > a)
		{
			Complex t = x[a];
			x[a] = x[b];
			x[b] = t;
		}
	}
}
\end{lstlisting}
\\
\bottomrule
\end{tabular}
\end{table}