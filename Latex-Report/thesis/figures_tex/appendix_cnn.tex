\definecolor{highlight_red}{rgb}{1.0,0.92,0.8}
\definecolor{highlight_yellow}{rgb}{1.0,1.0,0.75}
\definecolor{highlight_green}{rgb}{0.75,1.0,0.75}
\newcommand{\Hilight}{\makebox[0pt][l]{\color{highlight_yellow}\rule[-4pt]{0.78\linewidth}{14pt}}}
\newcommand{\hl}{\makebox[0pt][l]{\color{highlight_red}\rule[-4pt]{0.18\linewidth}{12pt}}}
\newcommand{\hlg}{\makebox[0pt][l]{\color{highlight_green}\rule[-4pt]{0.3\linewidth}{12pt}}}

\lstset { %
	language=C,
	backgroundcolor=\color{white},
	basicstyle=\ttfamily\tiny,
	keywordstyle=\color{magenta}\ttfamily,
	stringstyle=\color{blue}\ttfamily,
	commentstyle=\color{darkgreen}\ttfamily,
    breakatwhitespace=false,
	breaklines=true,
    showstringspaces=false
}
\lstset{framesep=-5pt, xleftmargin=-5pt}
\begin{table}[!h]
\centering
\caption{Removal of Altera device-specific Macros}
\label{cnncode1:altera-dep-removal}
\begin{tabular}{l}
\toprule
\begin{lstlisting}[columns=fullflexible, language=C++, escapechar=\$]
void createDeviceBuffer() {
	cl_int status;
	cout << "Allocating buffers on the device memory" << endl;
	// data is allocated in BANK1 and weights are in BANK2 for efficient access.
	d_input_img = clCreateBuffer(context, CL_MEM_READ_ONLY |$\hl$ CL_MEM_BANK_1_ALTERA, 
		conv1.bot_shape->x * conv1.bot_shape->y  * conv1.bot_shape->z * sizeof(DTYPE), NULL, &status);
        conv1.d_input = clCreateBuffer(context, CL_MEM_WRITE_ONLY | $\hl$ CL_MEM_BANK_1_ALTERA,
		(conv1.bot_shape->x+2*conv1.pad) * (conv1.bot_shape->y+2*conv1.pad) * conv1.bot_shape->z * sizeof(DTYPE), NULL, &status);
        conv1.d_output = clCreateBuffer(context, CL_MEM_WRITE_ONLY | $\hl$ CL_MEM_BANK_1_ALTERA, 
		conv1.top_shape.x * conv1.top_shape.y  * conv1.top_shape.z * sizeof(DTYPE), NULL, &status);
	
	conv1.d_W = clCreateBuffer(context, CL_MEM_READ_ONLY | $\hl$ CL_MEM_BANK_2_ALTERA | CL_MEM_COPY_HOST_PTR, 
		conv1.K * conv1.K * conv1.bot_shape->z * conv1.top_shape.z * sizeof(WTYPE), conv1.W, &status);
	conv1.d_b = clCreateBuffer(context, CL_MEM_READ_ONLY | $\hl$ CL_MEM_BANK_2_ALTERA | CL_MEM_COPY_HOST_PTR, 
		conv1.top_shape.z * sizeof(WTYPE), conv1.b, &status);
        ........
}
\end{lstlisting}
\\
\bottomrule
\end{tabular}
\end{table}

\begin{table}[!h]
\centering
\caption{Loading kernel from Source}
\label{cnncode2:load-kernel-source}
\begin{tabular}{l}
\toprule
\begin{lstlisting}[columns=fullflexible, language=C++]
long LoadOpenCLKernel(char const* path, char **buf)
{
    FILE  *fp;
    size_t fsz;
    long   off_end;
    int    rc;
    /* Open the file */
    fp = fopen(path, "r");
    if( NULL == fp ) {
        return -1L;
    }
    /* Seek to the end of the file */
    rc = fseek(fp, 0L, SEEK_END);
    if( 0 != rc ) {
        return -1L;
    }
    /* Byte offset to the end of the file (size) */
    if( 0 > (off_end = ftell(fp)) ) {
        return -1L;
    }
    fsz = (size_t)off_end;
    /* Allocate a buffer to hold the whole file */
    *buf = (char *) malloc( fsz+1);
    if( NULL == *buf ) {
        return -1L;
    }
    /* Rewind file pointer to start of file */
    rewind(fp);
    /* Slurp file into buffer */
    if( fsz != fread(*buf, 1, fsz, fp) ) {
        free(*buf);
        return -1L;
    }
    /* Close the file */
    if( EOF == fclose(fp) ) {
        free(*buf);
        return -1L;
    }
    /* Make sure the buffer is NUL-terminated, just in case */
    (*buf)[fsz] = '\0';
    /* Return the file size */
    return (long)fsz;
}
\end{lstlisting}
\\
\bottomrule
\end{tabular}
\end{table}

\begin{table}[!h]
\centering
\caption{Initialization of OpenCL Objects for Altera FPGA (Taken from Altera Design Examples)}
\label{cnncode3:altera-opencl-init}
\begin{tabular}{l}
\toprule
\begin{lstlisting}[columns=fullflexible, language=C++, escapechar=\$]
bool init_opencl() {
	cl_int status;	
	printf("Initializing OpenCL\n");
	if(!setCwdToExeDir()) {
	  return false;
	}	
	// Get the OpenCL platform.
	$\Hilight$ platform = findPlatform("Altera");
	 if(platform == NULL) {
	  printf("ERROR: Unable to find Altera OpenCL platform.\n");
	 return false;
	}	
	// Query the available OpenCL device.
	$\Hilight$ devices.reset(getDevices(platform, CL_DEVICE_TYPE_ALL, &num_devices));
	printf("Platform: %s\n", getPlatformName(platform).c_str());
	printf("Found %d devices in the board. Using only one device for this app\n", num_devices);
	for(unsigned i = 0; i < num_devices; ++i) {
	  printf("  %s\n", getDeviceName(devices[i]).c_str());
	}
	$\Hilight$ target_device = devices[0];	
	// Create the context.
	context = clCreateContext(NULL, num_devices, &target_device, &oclContextCallback, NULL, &status);
	checkError(status, "Failed to create context");
	
	std::string binary_file = getBoardBinaryFile("cnn_kernels", target_device);
	printf("Using AOCX: %s\n", binary_file.c_str());
	$\Hilight$ program = createProgramFromBinary(context, binary_file.c_str(), &target_device, num_devices);
	
	// Build the program that was just created.
	status = clBuildProgram(program, 0, NULL, "", NULL, NULL);
	checkError(status, "Failed to build program");
	
	kernel.reset(num_kernels);
	
	// Command queue.
	queue = clCreateCommandQueue(context, target_device, CL_QUEUE_PROFILING_ENABLE, &status);
	checkError(status, "Failed to create command queue");
	
	// Kernel.
	kernel[0] = clCreateKernel(program, "filter3D", &status);
	checkError(status, "Failed to create kernel");
	kernel[1] = clCreateKernel(program, "maxpool3D", &status);
	checkError(status, "Failed to create kernel");
	kernel[2] = clCreateKernel(program, "iplayer", &status);
	checkError(status, "Failed to create kernel");
	kernel[3] = clCreateKernel(program, "relu_layer", &status);
	checkError(status, "Failed to create kernel");
	kernel[4] = clCreateKernel(program, "softmax", &status);
	checkError(status, "Failed to create kernel");

	return true;
}
\end{lstlisting}
\\
\bottomrule
\end{tabular}
\end{table}

\begin{table}[!h]
\centering
\caption{Initialization of OpenCL Objects for a GPU Device}
\label{cnncode4:gpu-opencl-init}
\begin{tabular}{l}
\toprule
\begin{lstlisting}[columns=fullflexible, language=C++,escapechar=\$]
bool init_opencl() {
	cl_int status;
	int err;
	cl_platform_id platform_ids[5];		
	char *KernelSource;
	long lFileSize;
	cl_uint dev_cnt = 0;

        printf("Initializing OpenCL\n");
	clGetPlatformIDs(0, 0, &dev_cnt);
	$\Hilight$ clGetPlatformIDs(dev_cnt, platform_ids, NULL);
        int gpu = 1;
	for(unsigned i = 0;i < dev_cnt; i++)
	{
	   err = clGetDeviceIDs(platform_ids[i], $\hlg$ gpu ? CL_DEVICE_TYPE_GPU:CL_DEVICE_TYPE_CPU, 1, &target_device, NULL);
           if(err == CL_SUCCESS)
           {
		break;
	   }
        }
	if (err != CL_SUCCESS)
	{
	   printf("Error: Failed to create a device group!\n");
	   return EXIT_FAILURE;
   	}

	// Create the context.
        context = clCreateContext(0, 1, &target_device, NULL, NULL, &err);
	if (!context)
	{
	    printf("Error: Failed to create a compute context!\n");
	    return EXIT_FAILURE;
	}

	$\Hilight$ lFileSize = LoadOpenCLKernel("device/cnn_kernels.cl", &KernelSource);
	if( lFileSize < 0L ) {
		perror("File read failed");
		return 1;
	}

	$\Hilight$ program = clCreateProgramWithSource(context, 1, (const char **) & KernelSource, NULL, &err);
	if (!program)
	{
		printf("Error: Failed to create compute program!\n");
		return EXIT_FAILURE;
	}

	// Build the program that was just created.
	$\Hilight$ status = clBuildProgram(program, 0, NULL, "", NULL, NULL);
	checkError(status, "Failed to build program");
	
	kernel.reset(num_kernels);
	// Command queue.
	$\Hilight$ queue = clCreateCommandQueue(context, target_device, CL_QUEUE_PROFILING_ENABLE, &status); 
	checkError(status, "Failed to create command queue");
	
	// Kernel.
	kernel[0] = clCreateKernel(program, "filter3D", &status);
	checkError(status, "Failed to create kernel");
	kernel[1] = clCreateKernel(program, "maxpool3D", &status);
	checkError(status, "Failed to create kernel");
	kernel[2] = clCreateKernel(program, "iplayer", &status);
	checkError(status, "Failed to create kernel");
	kernel[3] = clCreateKernel(program, "relu_layer", &status);
	checkError(status, "Failed to create kernel");
	kernel[4] = clCreateKernel(program, "softmax", &status);
	checkError(status, "Failed to create kernel");
	return true;
}
\end{lstlisting}
\\
\bottomrule
\end{tabular}
\end{table}